\documentclass[12pt]{article}

\usepackage{geometry}
\setlength{\parindent}{0pt}
% \setlength{\textwidth}{6.5in}
\setlength{\textwidth}{7.2in}

% \usepackage{gchords2}
\usepackage{gchords}

\iftrue
% From: Richard Cobbe (cobbe@directlink.net)
% Subject: Re: Raised Guitar Chords in LaTeX 
% Newsgroups: comp.text.tex
% Date: 2000/04/20 
% \newcommand{\chord}...

\newcommand\mychords{
\def\chordsize{2.2mm}   % distance between two frets (and two strings)
\def\numfrets{6}
\font\fingerfont=cmr5  % font used for numbering fingers
% \font\fingerfont=cmmi5   % font used for numbering fingers
\font\namefont=cmr12    % font used for labeling of the chord
\font\fretposfont=cmr7  % font used for the fret position marker
% \def\dampsymbol{$\scriptstyle\times$} %  `damp this string' marker
\def\dampsymbol{{\tiny$\scriptstyle\times$}} %  `damp this string' marker
}

\renewcommand\yoff{3}
\renewcommand\fingsiz{1.6}

% corcovado:

\newcommand{\AmiSeven}{\chord{t}{f1p5,x,f2p5,f3p5,f4p5,x}{Ami${\,}^{7}$}}
\newcommand{\AmiSix}{\chord{t}{f2p5,x,f1p4,f3p5,f4p5,x}{Ami${\,}^{6}$}}
\newcommand{\DNine}{\chord{t}{x,p5,p4,p5,p5,x}{D${\,}^{9}$}}
\newcommand{\ENineMinusGSharp}{\chord{t}{p4,x,p3,p4,p5,x}{E${\,}^{9-}$/G${}^{\#}$}}
\newcommand{\CmiSevenFiveMinus}{\chord{t}{x,p3,p4,p3,p4,x}{Cmi${\,}^{7}_{5-}$}}

\newcommand{\GmiSeven}{\chord{t}{f1p3,x,f2p3,f3p3,f4p3,x}{Gmi${\,}^{7}$}}
\newcommand{\CNine}{\chord{t}{x,p3,p2,p3,p3,x}{C${\,}^{9}$}}
\newcommand{\DNineMinusFSharp}{\chord{t}{p2,x,p1,p2,p3,x}{D${\,}^{9-}$/F${}^{\#}$}}
\newcommand{\FSevenMaj}{\chord{t}{p1,x,p2,p2,p1,x}{F${\,}^{7{\rm maj}}$}}

\newcommand{\FmiSeven}{\chord{t}{p1,x,p1,p1,p1,x}{Fmi${\,}^{7}$}}
\newcommand{\BNine}{\chord{t}{x,p1,n,p1,p1,x}{B${}^{{\rm b}\,9}$}}
\newcommand{\ASevenFivePlus}{\chord{t}{n,n,p3,n,p2,n}{A${\,}^{7}_{5+}$}}
\newcommand{\ASeven}{\chord{t}{n,n,p2,n,p2,n}{A${\,}^{7}$}}
\newcommand{\DmiSeven}{\chord{t}{x,n,n,p2,p1,p1}{Dmi${\,}^{7}$}}
\newcommand{\EmiSeven}{\chord{t}{n,2,n,n,p3,n}{Emi${\,}^{7}$}}
\newcommand{\EmiSevenFiveMinus}{\chord{t}{n,x,p2,p3,p3,n}{Emi${\,}^{7}_{5-}$}}

\fi

\begin{document}

\mychords

% \mediumchords
% \chord{t}{x,f3p3,f2p2,n,f1p1,f4p3,}{C}
\pagestyle{empty}

% textwidth is \textwidth.
\begin{center}
\Large
\textbf{Corcovado} \\ \textsl{Antonio Carlos Jobim} \\
 \textsl{} \\
\end{center}

{ \large

\begin{verse}

\upchord{\AmiSix}Quiet nights of quiet stars \upchord{\DNine}
\\ \upchord{\ENineMinusGSharp} Quiet chords from my guitar \upchord{\CmiSevenFiveMinus}
\\ \upchord{\GmiSeven}Floating on the \upchord{\CNine}silence \upchord{\DNineMinusFSharp}that sur\upchord{\FSevenMaj}rounds us.
\end{verse}

\begin{verse}
\upchord{\FmiSeven}   Quiet thoughts and quiet dreams\upchord{\BNine}
\\ \upchord{\ASevenFivePlus}Quiet walks by quiet streams
\\ \upchord{\AmiSeven}and a window that looks out on \upchord{\DmiSeven}Corcovado
% looking on the mountains and the sea.
\\ Oh, how \upchord{\FmiSeven}lovely.
\end{verse}

\begin{verse}
 This is where I want to be
\\ here with you so close to me
\\ until the final flicker of life's ember.
\end{verse}

\begin{verse}
   \upchord{\FmiSeven}I who was lost and \upchord{\BNine}lonely
\\ \upchord{\EmiSeven}believeing life was \upchord{\AmiSeven}only 
\\ \upchord{\DmiSeven}a bitter tragic \upchord{\FmiSeven}joke I found with \upchord{\EmiSevenFiveMinus}you.{\hspace{1cm}} \upchord{\ASevenFivePlus\ASeven}
% \\ oh my dear.
\\ \upchord{\DmiSeven}the meaning of \upchord{\FmiSeven}existence.
\\ Oh, my \upchord{\AmiSix}love.
\end{verse}

% \pagebreak

\begin{verse}
   Um cantinho, um viol\~{a}o
\\ Esse amor, uma can{\c{c}}\~{a}o
\\ Pra fazer feliz a quem se ama
\end{verse}

\begin{verse}
   Muita calma pra pensar
\\ E ter tempo pra sonhar
\\ Da janela v\^{e}-se o Corcovado
\\ O Redentor, que lindo!
\end{verse}

\begin{verse}
   Quero a vida sempre assim
\\ Com voc\^{e} perto de mim
\\ At\'{e} o apagar da velha chama
\end{verse}

\begin{verse}
   E eu que era triste
\\ Descrente desse mundo
\\ Ao encontrar voc\^{e} eu conheci
\\ O que \'{e} felicidade, meu amor.
\end{verse}

}

\newpage
Note: E${\,}^{9-}$/G${}^{\#}$ can be also called G${\,}^{13}$/9-.

\end{document}

